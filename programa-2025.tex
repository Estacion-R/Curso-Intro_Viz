% Options for packages loaded elsewhere
\PassOptionsToPackage{unicode}{hyperref}
\PassOptionsToPackage{hyphens}{url}
\PassOptionsToPackage{dvipsnames,svgnames,x11names}{xcolor}
%
\documentclass[
  11pt,
  letterpaper,
  DIV=11,
  numbers=noendperiod]{scrartcl}

\usepackage{amsmath,amssymb}
\usepackage{iftex}
\ifPDFTeX
  \usepackage[T1]{fontenc}
  \usepackage[utf8]{inputenc}
  \usepackage{textcomp} % provide euro and other symbols
\else % if luatex or xetex
  \usepackage{unicode-math}
  \defaultfontfeatures{Scale=MatchLowercase}
  \defaultfontfeatures[\rmfamily]{Ligatures=TeX,Scale=1}
\fi
\usepackage{lmodern}
\ifPDFTeX\else  
    % xetex/luatex font selection
    \setmainfont[]{Ubuntu}
\fi
% Use upquote if available, for straight quotes in verbatim environments
\IfFileExists{upquote.sty}{\usepackage{upquote}}{}
\IfFileExists{microtype.sty}{% use microtype if available
  \usepackage[]{microtype}
  \UseMicrotypeSet[protrusion]{basicmath} % disable protrusion for tt fonts
}{}
\makeatletter
\@ifundefined{KOMAClassName}{% if non-KOMA class
  \IfFileExists{parskip.sty}{%
    \usepackage{parskip}
  }{% else
    \setlength{\parindent}{0pt}
    \setlength{\parskip}{6pt plus 2pt minus 1pt}}
}{% if KOMA class
  \KOMAoptions{parskip=half}}
\makeatother
\usepackage{xcolor}
\setlength{\emergencystretch}{3em} % prevent overfull lines
\setcounter{secnumdepth}{-\maxdimen} % remove section numbering
% Make \paragraph and \subparagraph free-standing
\makeatletter
\ifx\paragraph\undefined\else
  \let\oldparagraph\paragraph
  \renewcommand{\paragraph}{
    \@ifstar
      \xxxParagraphStar
      \xxxParagraphNoStar
  }
  \newcommand{\xxxParagraphStar}[1]{\oldparagraph*{#1}\mbox{}}
  \newcommand{\xxxParagraphNoStar}[1]{\oldparagraph{#1}\mbox{}}
\fi
\ifx\subparagraph\undefined\else
  \let\oldsubparagraph\subparagraph
  \renewcommand{\subparagraph}{
    \@ifstar
      \xxxSubParagraphStar
      \xxxSubParagraphNoStar
  }
  \newcommand{\xxxSubParagraphStar}[1]{\oldsubparagraph*{#1}\mbox{}}
  \newcommand{\xxxSubParagraphNoStar}[1]{\oldsubparagraph{#1}\mbox{}}
\fi
\makeatother


\providecommand{\tightlist}{%
  \setlength{\itemsep}{0pt}\setlength{\parskip}{0pt}}\usepackage{longtable,booktabs,array}
\usepackage{calc} % for calculating minipage widths
% Correct order of tables after \paragraph or \subparagraph
\usepackage{etoolbox}
\makeatletter
\patchcmd\longtable{\par}{\if@noskipsec\mbox{}\fi\par}{}{}
\makeatother
% Allow footnotes in longtable head/foot
\IfFileExists{footnotehyper.sty}{\usepackage{footnotehyper}}{\usepackage{footnote}}
\makesavenoteenv{longtable}
\usepackage{graphicx}
\makeatletter
\newsavebox\pandoc@box
\newcommand*\pandocbounded[1]{% scales image to fit in text height/width
  \sbox\pandoc@box{#1}%
  \Gscale@div\@tempa{\textheight}{\dimexpr\ht\pandoc@box+\dp\pandoc@box\relax}%
  \Gscale@div\@tempb{\linewidth}{\wd\pandoc@box}%
  \ifdim\@tempb\p@<\@tempa\p@\let\@tempa\@tempb\fi% select the smaller of both
  \ifdim\@tempa\p@<\p@\scalebox{\@tempa}{\usebox\pandoc@box}%
  \else\usebox{\pandoc@box}%
  \fi%
}
% Set default figure placement to htbp
\def\fps@figure{htbp}
\makeatother

\KOMAoption{captions}{tableheading}
\makeatletter
\@ifpackageloaded{caption}{}{\usepackage{caption}}
\AtBeginDocument{%
\ifdefined\contentsname
  \renewcommand*\contentsname{Table of contents}
\else
  \newcommand\contentsname{Table of contents}
\fi
\ifdefined\listfigurename
  \renewcommand*\listfigurename{List of Figures}
\else
  \newcommand\listfigurename{List of Figures}
\fi
\ifdefined\listtablename
  \renewcommand*\listtablename{List of Tables}
\else
  \newcommand\listtablename{List of Tables}
\fi
\ifdefined\figurename
  \renewcommand*\figurename{Figure}
\else
  \newcommand\figurename{Figure}
\fi
\ifdefined\tablename
  \renewcommand*\tablename{Table}
\else
  \newcommand\tablename{Table}
\fi
}
\@ifpackageloaded{float}{}{\usepackage{float}}
\floatstyle{ruled}
\@ifundefined{c@chapter}{\newfloat{codelisting}{h}{lop}}{\newfloat{codelisting}{h}{lop}[chapter]}
\floatname{codelisting}{Listing}
\newcommand*\listoflistings{\listof{codelisting}{List of Listings}}
\makeatother
\makeatletter
\makeatother
\makeatletter
\@ifpackageloaded{caption}{}{\usepackage{caption}}
\@ifpackageloaded{subcaption}{}{\usepackage{subcaption}}
\makeatother

\usepackage{bookmark}

\IfFileExists{xurl.sty}{\usepackage{xurl}}{} % add URL line breaks if available
\urlstyle{same} % disable monospaced font for URLs
\hypersetup{
  pdftitle={Taller: Visualización de datos con R y ggplot2},
  pdfauthor={Estación R},
  colorlinks=true,
  linkcolor={blue},
  filecolor={Maroon},
  citecolor={Blue},
  urlcolor={Blue},
  pdfcreator={LaTeX via pandoc}}


\title{Taller: Visualización de datos con R y ggplot2}
\author{Estación R}
\date{2025-07-01}

\begin{document}
\maketitle


\section{Programa del Taller}\label{programa-del-taller}

\textbf{Duración total:} 3 encuentros de 2 h cada uno\\
\textbf{Modalidad:} Teórico-práctica\\
\textbf{Nivel:} Personas con conocimientos básicos de R y tidyverse\\
\textbf{Formato del código:} R script y RMarkdown\\
\textbf{Datos utilizados:} Casos reales en Ciencias Sociales + datasets
generales

\subsection{Objetivos}\label{objetivos}

\begin{itemize}
\tightlist
\item
  Comprender los principios fundamentales de la visualización de datos.
\item
  Aprender a construir gráficos efectivos, reproducibles y estéticos
  usando \texttt{\{ggplot2\}}.
\item
  Conocer estrategias para mejorar la estética y el diseño gráfico.
\item
  Incorporar herramientas para generar gráficos interactivos y anotados.
\item
  Practicar con datos reales y aplicar lo aprendido en visualizaciones
  propias.
\end{itemize}

\begin{center}\rule{0.5\linewidth}{0.5pt}\end{center}

\subsection{Encuentro 1 --- Fundamentos de visualización con
ggplot2}\label{encuentro-1-fundamentos-de-visualizaciuxf3n-con-ggplot2}

Inspirado en \emph{The Carpentries} y \emph{Jenny Bryan}

\textbf{Contenidos:}

\begin{itemize}
\tightlist
\item
  Introducción a la visualización como herramienta de análisis y
  comunicación.
\item
  Gramática de los gráficos: \texttt{data\ +\ aesthetics\ +\ geoms}
\item
  Primeros pasos con \texttt{\{ggplot2\}}

  \begin{itemize}
  \tightlist
  \item
    \texttt{ggplot(data,\ aes(x,\ y))\ +\ geom\_*()}
  \item
    Gráficos de barras, puntos y líneas
  \end{itemize}
\item
  Tipos de variables (categóricas vs.~continuas)
\item
  Mapeo estético vs.~valores fijos
\item
  Temas y escalas básicas
\item
  Uso de \texttt{theme\_minimal()} y temas predefinidos
\end{itemize}

\textbf{Ejercicio práctico:}

\begin{itemize}
\tightlist
\item
  Crear 3 visualizaciones distintas con un mismo dataset (barra, línea,
  puntos).
\item
  Aplicar combinaciones básicas de \texttt{aes()} y \texttt{geom\_*()}.
\end{itemize}

\begin{center}\rule{0.5\linewidth}{0.5pt}\end{center}

\subsection{Encuentro 2 --- Personalización, diseño y
estética}\label{encuentro-2-personalizaciuxf3n-diseuxf1o-y-estuxe9tica}

Inspirado en \emph{Cédric Scherer} y \emph{Thomas Lin Pedersen}

\textbf{Contenidos:}

\begin{itemize}
\tightlist
\item
  Elección del tipo de gráfico según el tipo de variable
\item
  Personalización visual:

  \begin{itemize}
  \tightlist
  \item
    \texttt{labs()}, \texttt{theme()}, \texttt{scale\_*\_*()}
  \item
    Paletas con \texttt{\{RColorBrewer\}} y
    \texttt{scale\_fill\_manual()}
  \end{itemize}
\item
  Facetado con \texttt{facet\_wrap()} y \texttt{facet\_grid()}
\item
  Principios de diseño gráfico:

  \begin{itemize}
  \tightlist
  \item
    Contraste, jerarquía, alineación, claridad
  \item
    Proporciones, márgenes, tipografía
  \end{itemize}
\item
  Composición con \texttt{\{patchwork\}} para gráficos múltiples
\end{itemize}

\textbf{Ejercicio práctico:}

\begin{itemize}
\tightlist
\item
  Reproducir una visualización compleja y mejorar su diseño.
\item
  Aplicar principios de diseño visual sobre una visualización propia.
\end{itemize}

\begin{center}\rule{0.5\linewidth}{0.5pt}\end{center}

\subsection{Encuentro 3 --- Anotaciones, interactividad y
publicación}\label{encuentro-3-anotaciones-interactividad-y-publicaciuxf3n}

Inspirado en \emph{Thomas Lin Pedersen} y \emph{Cédric Scherer}

\textbf{Contenidos:}

\begin{itemize}
\tightlist
\item
  Anotaciones en gráficos:

  \begin{itemize}
  \tightlist
  \item
    \texttt{annotate()}, \texttt{geom\_text()}, \texttt{geom\_label()}
  \item
    Uso de \texttt{\{ggannotate\}} para exportar anotaciones visuales
  \end{itemize}
\item
  Exportación y publicación:

  \begin{itemize}
  \tightlist
  \item
    \texttt{ggsave()}, DPI, tamaños, proporciones
  \end{itemize}
\item
  Interactividad:

  \begin{itemize}
  \tightlist
  \item
    \texttt{ggplotly()} para transformar gráficos estáticos en
    interactivos
  \end{itemize}
\item
  Buenas prácticas para comunicación visual efectiva
\item
  Recursos para seguir aprendiendo
\end{itemize}

\textbf{Ejercicio práctico:}

\begin{itemize}
\tightlist
\item
  Anotar y exportar una visualización propia.
\item
  Compartir el resultado en formato interactivo o publicable.
\end{itemize}

\begin{center}\rule{0.5\linewidth}{0.5pt}\end{center}

\subsection{Paquetes principales
utilizados}\label{paquetes-principales-utilizados}

\begin{itemize}
\tightlist
\item
  \texttt{\{ggplot2\}}
\item
  \texttt{\{tidyverse\}}
\item
  \texttt{\{ggannotate\}}
\item
  \texttt{\{plotly\}}
\item
  \texttt{\{patchwork\}}
\item
  \texttt{\{RColorBrewer\}} / \texttt{\{viridis\}}
\item
  \texttt{\{ggtext\}} / \texttt{\{scales\}} (según necesidad)
\end{itemize}




\end{document}
